\documentclass{article}
\usepackage{amsmath}
\begin{document}

Calculate the spin-orbit splitting of the L-shell hole. With a reference state 
$|0\rangle$ the 2p-hole states span a six-dimensional subspace corresponding to the removal of a $2p$-electron ($2p_x$, $2p_y$, $2p_z$) of spin $\alpha$ or $\beta$
\begin{equation}
    |p_{i\sigma}^{-1}\rangle = a_{i\sigma}|0\rangle\qquad i=x,y,z \qquad\sigma=\alpha,\beta
\end{equation}
and the diagonalization of the spin-orbit operator in this basis leads for the atomic case to two degenerate sets of $J=1/2$ and $J=3/2$

\begin{equation}
 V^{\sigma\tau}_{ij} = \langle p_{i\sigma}^{-1}|V|p_{j\tau}^{-1}\rangle
                     = \langle 0 | a_{i\sigma}^\dagger V a_{j\tau} |0\rangle
\end{equation}
Explicit blocking of the spin part yields the matrix
\begin{equation}
 \mathbf{V} = \begin{pmatrix}
              V^{\alpha\alpha}_{ij} & V_{ij}^{\alpha\beta} \\
              V^{\beta\alpha}_{ij}  & V_{ij}^{\beta\beta} \\
            \end{pmatrix}
\end{equation}
where each subblock is a 3x3 matrix labeld by the $p$ orbitals.
Spin-orbit selectron roules give
\begin{eqnarray}
 V^{\alpha\alpha}_{ij} &=& \langle p_{i\alpha}^{-1}|l_z s_z |p_{j\alpha}^{-1}\rangle \\
 V^{\beta\beta}_{ij}   &=& \langle p_{i\beta}^{-1}|l_z s_z |p_{j\beta}^{-1}\rangle \\
 V^{\alpha\beta}_{ij}  &=& \langle p_{i\alpha}^{-1}|-l_{1} s_{-1} |p_{j\beta}^{-1}\rangle \\
 V^{\beta\alpha}_{ij}  &=& \langle p_{i\beta}^{-1}|-l_{-1} s_{1} |p_{j\alpha}^{-1}\rangle \\
\end{eqnarray}
and the off-diagonal blocks can be transformed to a common spin component with
the Wigner-Eckart theorem (notation $C^{j_1,j_2,j}_{m_1,m_2,m}$)
\begin{equation}
%V^{\alpha\beta}_{ij}  = \langle p_{i\alpha}^{-1}|-l_{1} s_{-1} |p_{j\beta}^{-1}\rangle \\
 V^{\alpha\beta}_{ij}  = \langle p_{i\beta }^{-1}|-l_{1} s_{0}  |p_{j\beta}^{-1}\rangle 
                         C^{\frac 1 2,1,\frac 1 2}_{\frac 1 2,-1,-\frac 1 2}/
                         C^{\frac 1 2,1,\frac 1 2}_{\frac 1 2, 0,\frac 1 2}
                       = \langle p_{i\beta }^{-1}|-l_{1} s_{0}  |p_{j\beta}^{-1}\rangle \sqrt{2/3}/\sqrt{1/3}
                       = \langle p_{i\beta }^{-1}|(l_x+il_y) s_z  |p_{j\beta}^{-1}\rangle
\end{equation}
\begin{equation}
%V^{\beta\alpha}_{ij}  = \langle p_{i\beta}^{-1}|-l_{-1} s_{1} |p_{j\alpha}^{-1}\rangle \\
 V^{\beta\alpha}_{ij}  = \langle p_{i\beta}^{-1}|-l_{-1} s_{0} |p_{j\beta }^{-1}\rangle 
                         C^{\frac 1 2,1,\frac 1 2}_{-\frac 1 2,1,\frac 1 2}/
                         C^{\frac 1 2,1,\frac 1 2}_{\frac 1 2, 0,\frac 1 2}
                       = \langle p_{i\beta}^{-1}|-l_{-1} s_{0} |p_{j\beta }^{-1}\rangle (-\sqrt{2/3})/\sqrt{1/3}
                       = \langle p_{i\beta}^{-1}|(l_x-il_y) s_z |p_{j\beta }^{-1}\rangle
\end{equation}
and finally 
\begin{equation}
%V^{\alpha\alpha}_{ij} &=& \langle p_{i\alpha}^{-1}|l_z s_z |p_{j\alpha}^{-1}\rangle \\
 V^{\alpha\alpha}_{ij}  =  \langle p_{i\beta}^{-1}|l_z s_z |p_{j\beta}^{-1}\rangle
                         C^{\frac 1 2,1,\frac 1 2}_{-\frac 1 2, 0,-\frac 1 2}/
                         C^{\frac 1 2,1,\frac 1 2}_{\frac 1 2, 0,\frac 1 2}
 = - \langle p_{i\beta}^{-1}|l_z s_z |p_{j\beta}^{-1}\rangle
\end{equation}

i.e.
\begin{equation}
 \mathbf{V}  =
    \langle 
        p_{\i\beta}^{-1}  |
         \begin{pmatrix} 
            -l_z s_z &(l_x + i l_y)s_z \\
            (l_x-il_y)s_z & l_z s_z
        \end{pmatrix}
        | p_{j\beta}^{-1}
    \rangle 
\end{equation}


\end{document}
